%% bare_jrnl.tex
%% V1.4b
%% 2015/08/26
%% by Michael Shell
%% see http://www.michaelshell.org/
%% for current contact information.
%%
%% This is a skeleton file demonstrating the use of IEEEtran.cls
%% (requires IEEEtran.cls version 1.8b or later) with an IEEE
%% journal paper.
%%
%% Support sites:
%% http://www.michaelshell.org/tex/ieeetran/
%% http://www.ctan.org/pkg/ieeetran
%% and
%% http://www.ieee.org/
\newcommand{\code}[1]{\texttt{\detokenize{#1}}}
%%*************************************************************************
%% Legal Notice:
%% This code is offered as-is without any warranty either expressed or
%% implied; without even the implied warranty of MERCHANTABILITY or
%% FITNESS FOR A PARTICULAR PURPOSE! 
%% User assumes all risk.
%% In no event shall the IEEE or any contributor to this code be liable for
%% any damages or losses, including, but not limited to, incidental,
%% consequential, or any other damages, resulting from the use or misuse
%% of any information contained here.
%%
%% All comments are the opinions of their respective authors and are not
%% necessarily endorsed by the IEEE.
%%
%% This work is distributed under the LaTeX Project Public License (LPPL)
%% ( http://www.latex-project.org/ ) version 1.3, and may be freely used,
%% distributed and modified. A copy of the LPPL, version 1.3, is included
%% in the base LaTeX documentation of all distributions of LaTeX released
%% 2003/12/01 or later.
%% Retain all contribution notices and credits.
%% ** Modified files should be clearly indicated as such, including  **
%% ** renaming them and changing author support contact information. **
%%*************************************************************************


% *** Authors should verify (and, if needed, correct) their LaTeX system  ***
% *** with the testflow diagnostic prior to trusting their LaTeX platform ***
% *** with production work. The IEEE's font choices and paper sizes can   ***
% *** trigger bugs that do not appear when using other class files.       ***                          ***
% The testflow support page is at:
% http://www.michaelshell.org/tex/testflow/



\documentclass[journal]{IEEEtran}
\usepackage{url}
%
% If IEEEtran.cls has not been installed into the LaTeX system files,
% manually specify the path to it like:
% \documentclass[journal]{../sty/IEEEtran}





% Some very useful LaTeX packages include:
% (uncomment the ones you want to load)


% *** MISC UTILITY PACKAGES ***
%
%\usepackage{ifpdf}
% Heiko Oberdiek's ifpdf.sty is very useful if you need conditional
% compilation based on whether the output is pdf or dvi.
% usage:
% \ifpdf
%   % pdf code
% \else
%   % dvi code
% \fi
% The latest version of ifpdf.sty can be obtained from:
% http://www.ctan.org/pkg/ifpdf
% Also, note that IEEEtran.cls V1.7 and later provides a builtin
% \ifCLASSINFOpdf conditional that works the same way.
% When switching from latex to pdflatex and vice-versa, the compiler may
% have to be run twice to clear warning/error messages.






% *** CITATION PACKAGES ***
%
%\usepackage{cite}
% cite.sty was written by Donald Arseneau
% V1.6 and later of IEEEtran pre-defines the format of the cite.sty package
% \cite{} output to follow that of the IEEE. Loading the cite package will
% result in citation numbers being automatically sorted and properly
% "compressed/ranged". e.g., [1], [9], [2], [7], [5], [6] without using
% cite.sty will become [1], [2], [5]--[7], [9] using cite.sty. cite.sty's
% \cite will automatically add leading space, if needed. Use cite.sty's
% noadjust option (cite.sty V3.8 and later) if you want to turn this off
% such as if a citation ever needs to be enclosed in parenthesis.
% cite.sty is already installed on most LaTeX systems. Be sure and use
% version 5.0 (2009-03-20) and later if using hyperref.sty.
% The latest version can be obtained at:
% http://www.ctan.org/pkg/cite
% The documentation is contained in the cite.sty file itself.






% *** GRAPHICS RELATED PACKAGES ***
%
\ifCLASSINFOpdf
  % \usepackage[pdftex]{graphicx}
  % declare the path(s) where your graphic files are
  % \graphicspath{{../pdf/}{../jpeg/}}
  % and their extensions so you won't have to specify these with
  % every instance of \includegraphics
  % \DeclareGraphicsExtensions{.pdf,.jpeg,.png}
\else
  % or other class option (dvipsone, dvipdf, if not using dvips). graphicx
  % will default to the driver specified in the system graphics.cfg if no
  % driver is specified.
  % \usepackage[dvips]{graphicx}
  % declare the path(s) where your graphic files are
  % \graphicspath{{../eps/}}
  % and their extensions so you won't have to specify these with
  % every instance of \includegraphics
  % \DeclareGraphicsExtensions{.eps}
\fi
% graphicx was written by David Carlisle and Sebastian Rahtz. It is
% required if you want graphics, photos, etc. graphicx.sty is already
% installed on most LaTeX systems. The latest version and documentation
% can be obtained at: 
% http://www.ctan.org/pkg/graphicx
% Another good source of documentation is "Using Imported Graphics in
% LaTeX2e" by Keith Reckdahl which can be found at:
% http://www.ctan.org/pkg/epslatex
%
% latex, and pdflatex in dvi mode, support graphics in encapsulated
% postscript (.eps) format. pdflatex in pdf mode supports graphics
% in .pdf, .jpeg, .png and .mps (metapost) formats. Users should ensure
% that all non-photo figures use a vector format (.eps, .pdf, .mps) and
% not a bitmapped formats (.jpeg, .png). The IEEE frowns on bitmapped formats
% which can result in "jaggedy"/blurry rendering of lines and letters as
% well as large increases in file sizes.
%
% You can find documentation about the pdfTeX application at:
% http://www.tug.org/applications/pdftex





% *** MATH PACKAGES ***
%
%\usepackage{amsmath}
% A popular package from the American Mathematical Society that provides
% many useful and powerful commands for dealing with mathematics.
%
% Note that the amsmath package sets \interdisplaylinepenalty to 10000
% thus preventing page breaks from occurring within multiline equations. Use:
%\interdisplaylinepenalty=2500
% after loading amsmath to restore such page breaks as IEEEtran.cls normally
% does. amsmath.sty is already installed on most LaTeX systems. The latest
% version and documentation can be obtained at:
% http://www.ctan.org/pkg/amsmath





% *** SPECIALIZED LIST PACKAGES ***
%
%\usepackage{algorithmic}
% algorithmic.sty was written by Peter Williams and Rogerio Brito.
% This package provides an algorithmic environment fo describing algorithms.
% You can use the algorithmic environment in-text or within a figure
% environment to provide for a floating algorithm. Do NOT use the algorithm
% floating environment provided by algorithm.sty (by the same authors) or
% algorithm2e.sty (by Christophe Fiorio) as the IEEE does not use dedicated
% algorithm float types and packages that provide these will not provide
% correct IEEE style captions. The latest version and documentation of
% algorithmic.sty can be obtained at:
% http://www.ctan.org/pkg/algorithms
% Also of interest may be the (relatively newer and more customizable)
% algorithmicx.sty package by Szasz Janos:
% http://www.ctan.org/pkg/algorithmicx




% *** ALIGNMENT PACKAGES ***
%
%\usepackage{array}
% Frank Mittelbach's and David Carlisle's array.sty patches and improves
% the standard LaTeX2e array and tabular environments to provide better
% appearance and additional user controls. As the default LaTeX2e table
% generation code is lacking to the point of almost being broken with
% respect to the quality of the end results, all users are strongly
% advised to use an enhanced (at the very least that provided by array.sty)
% set of table tools. array.sty is already installed on most systems. The
% latest version and documentation can be obtained at:
% http://www.ctan.org/pkg/array


% IEEEtran contains the IEEEeqnarray family of commands that can be used to
% generate multiline equations as well as matrices, tables, etc., of high
% quality.




% *** SUBFIGURE PACKAGES ***
%\ifCLASSOPTIONcompsoc
%  \usepackage[caption=false,font=normalsize,labelfont=sf,textfont=sf]{subfig}
%\else
%  \usepackage[caption=false,font=footnotesize]{subfig}
%\fi
% subfig.sty, written by Steven Douglas Cochran, is the modern replacement
% for subfigure.sty, the latter of which is no longer maintained and is
% incompatible with some LaTeX packages including fixltx2e. However,
% subfig.sty requires and automatically loads Axel Sommerfeldt's caption.sty
% which will override IEEEtran.cls' handling of captions and this will result
% in non-IEEE style figure/table captions. To prevent this problem, be sure
% and invoke subfig.sty's "caption=false" package option (available since
% subfig.sty version 1.3, 2005/06/28) as this is will preserve IEEEtran.cls
% handling of captions.
% Note that the Computer Society format requires a larger sans serif font
% than the serif footnote size font used in traditional IEEE formatting
% and thus the need to invoke different subfig.sty package options depending
% on whether compsoc mode has been enabled.
%
% The latest version and documentation of subfig.sty can be obtained at:
% http://www.ctan.org/pkg/subfig




% *** FLOAT PACKAGES ***
%
%\usepackage{fixltx2e}
% fixltx2e, the successor to the earlier fix2col.sty, was written by
% Frank Mittelbach and David Carlisle. This package corrects a few problems
% in the LaTeX2e kernel, the most notable of which is that in current
% LaTeX2e releases, the ordering of single and double column floats is not
% guaranteed to be preserved. Thus, an unpatched LaTeX2e can allow a
% single column figure to be placed prior to an earlier double column
% figure.
% Be aware that LaTeX2e kernels dated 2015 and later have fixltx2e.sty's
% corrections already built into the system in which case a warning will
% be issued if an attempt is made to load fixltx2e.sty as it is no longer
% needed.
% The latest version and documentation can be found at:
% http://www.ctan.org/pkg/fixltx2e


%\usepackage{stfloats}
% stfloats.sty was written by Sigitas Tolusis. This package gives LaTeX2e
% the ability to do double column floats at the bottom of the page as well
% as the top. (e.g., "\begin{figure*}[!b]" is not normally possible in
% LaTeX2e). It also provides a command:
%\fnbelowfloat
% to enable the placement of footnotes below bottom floats (the standard
% LaTeX2e kernel puts them above bottom floats). This is an invasive package
% which rewrites many portions of the LaTeX2e float routines. It may not work
% with other packages that modify the LaTeX2e float routines. The latest
% version and documentation can be obtained at:
% http://www.ctan.org/pkg/stfloats
% Do not use the stfloats baselinefloat ability as the IEEE does not allow
% \baselineskip to stretch. Authors submitting work to the IEEE should note
% that the IEEE rarely uses double column equations and that authors should try
% to avoid such use. Do not be tempted to use the cuted.sty or midfloat.sty
% packages (also by Sigitas Tolusis) as the IEEE does not format its papers in
% such ways.
% Do not attempt to use stfloats with fixltx2e as they are incompatible.
% Instead, use Morten Hogholm'a dblfloatfix which combines the features
% of both fixltx2e and stfloats:
%
% \usepackage{dblfloatfix}
% The latest version can be found at:
% http://www.ctan.org/pkg/dblfloatfix




%\ifCLASSOPTIONcaptionsoff
%  \usepackage[nomarkers]{endfloat}
% \let\MYoriglatexcaption\caption
% \renewcommand{\caption}[2][\relax]{\MYoriglatexcaption[#2]{#2}}
%\fi
% endfloat.sty was written by James Darrell McCauley, Jeff Goldberg and 
% Axel Sommerfeldt. This package may be useful when used in conjunction with 
% IEEEtran.cls'  captionsoff option. Some IEEE journals/societies require that
% submissions have lists of figures/tables at the end of the paper and that
% figures/tables without any captions are placed on a page by themselves at
% the end of the document. If needed, the draftcls IEEEtran class option or
% \CLASSINPUTbaselinestretch interface can be used to increase the line
% spacing as well. Be sure and use the nomarkers option of endfloat to
% prevent endfloat from "marking" where the figures would have been placed
% in the text. The two hack lines of code above are a slight modification of
% that suggested by in the endfloat docs (section 8.4.1) to ensure that
% the full captions always appear in the list of figures/tables - even if
% the user used the short optional argument of \caption[]{}.
% IEEE papers do not typically make use of \caption[]'s optional argument,
% so this should not be an issue. A similar trick can be used to disable
% captions of packages such as subfig.sty that lack options to turn off
% the subcaptions:
% For subfig.sty:
% \let\MYorigsubfloat\subfloat
% \renewcommand{\subfloat}[2][\relax]{\MYorigsubfloat[]{#2}}
% However, the above trick will not work if both optional arguments of
% the \subfloat command are used. Furthermore, there needs to be a
% description of each subfigure *somewhere* and endfloat does not add
% subfigure captions to its list of figures. Thus, the best approach is to
% avoid the use of subfigure captions (many IEEE journals avoid them anyway)
% and instead reference/explain all the subfigures within the main caption.
% The latest version of endfloat.sty and its documentation can obtained at:
% http://www.ctan.org/pkg/endfloat
%
% The IEEEtran \ifCLASSOPTIONcaptionsoff conditional can also be used
% later in the document, say, to conditionally put the References on a 
% page by themselves.




% *** PDF, URL AND HYPERLINK PACKAGES ***
%
%\usepackage{url}
% url.sty was written by Donald Arseneau. It provides better support for
% handling and breaking URLs. url.sty is already installed on most LaTeX
% systems. The latest version and documentation can be obtained at:
% http://www.ctan.org/pkg/url
% Basically, \url{my_url_here}.




% *** Do not adjust lengths that control margins, column widths, etc. ***
% *** Do not use packages that alter fonts (such as pslatex).         ***
% There should be no need to do such things with IEEEtran.cls V1.6 and later.
% (Unless specifically asked to do so by the journal or conference you plan
% to submit to, of course. )


% correct bad hyphenation here
\hyphenation{op-tical net-works semi-conduc-tor}


\begin{document}
%
% paper title
% Titles are generally capitalized except for words such as a, an, and, as,
% at, but, by, for, in, nor, of, on, or, the, to and up, which are usually
% not capitalized unless they are the first or last word of the title.
% Linebreaks \\ can be used within to get better formatting as desired.
% Do not put math or special symbols in the title.
\title{Project Proposal for DS 340W}
%
%
% author names and IEEE memberships
% note positions of commas and nonbreaking spaces ( ~ ) LaTeX will not break
% a structure at a ~ so this keeps an author's name from being broken across
% two lines.
% use \thanks{} to gain access to the first footnote area
% a separate \thanks must be used for each paragraph as LaTeX2e's \thanks
% was not built to handle multiple paragraphs
%

\author{Logan~Camacho,
        Sam~Adebayo,
        and~Jacob~Gavin}% <-this % stops a space}


% note the % following the last \IEEEmembership and also \thanks - 
% these prevent an unwanted space from occurring between the last author name
% and the end of the author line. i.e., if you had this:
% 
% \author{....lastname \thanks{...} \thanks{...} }
%                     ^------------^------------^----Do not want these spaces!
%
% a space would be appended to the last name and could cause every name on that
% line to be shifted left slightly. This is one of those "LaTeX things". For
% instance, "\textbf{A} \textbf{B}" will typeset as "A B" not "AB". To get
% "AB" then you have to do: "\textbf{A}\textbf{B}"
% \thanks is no different in this regard, so shield the last } of each \thanks
% that ends a line with a % and do not let a space in before the next \thanks.
% Spaces after \IEEEmembership other than the last one are OK (and needed) as
% you are supposed to have spaces between the names. For what it is worth,
% this is a minor point as most people would not even notice if the said evil
% space somehow managed to creep in.



% The paper headers
\markboth{DS 340W Project Proposal, September~2025}%
{Shell \MakeLowercase{\textit{et al.}}: Bare Demo of IEEEtran.cls for IEEE Journals}
% The only time the second header will appear is for the odd numbered pages
% after the title page when using the twoside option.
% 
% *** Note that you probably will NOT want to include the author's ***
% *** name in the headers of peer review papers.                   ***
% You can use \ifCLASSOPTIONpeerreview for conditional compilation here if
% you desire.




% If you want to put a publisher's ID mark on the page you can do it like
% this:
%\IEEEpubid{0000--0000/00\$00.00~\copyright~2015 IEEE}
% Remember, if you use this you must call \IEEEpubidadjcol in the second
% column for its text to clear the IEEEpubid mark.



% use for special paper notices
%\IEEEspecialpapernotice{(Invited Paper)}




% make the title area
\maketitle

% As a general rule, do not put math, special symbols or citations
% in the abstract or keywords.
\begin{abstract}
(Logan Camacho) This project intends to pinpoint road segments within the state of Pennsylvania that carry a significantly higher proportion of car accidents, with predictions based on past car crash data in conjunction with mapped out traffic volume data.
\end{abstract}

% Note that keywords are not normally used for peerreview papers.
%\begin{IEEEkeywords}

%\end{IEEEkeywords}
%





% For peer review papers, you can put extra information on the cover
% page as needed:
% \ifCLASSOPTIONpeerreview
% \begin{center} \bfseries EDICS Category: 3-BBND \end{center}
% \fi
%
% For peerreview papers, this IEEEtran command inserts a page break and
% creates the second title. It will be ignored for other modes.
\IEEEpeerreviewmaketitle



\section{Introduction (Logan Camacho)}
% The very first letter is a 2 line initial drop letter followed
% by the rest of the first word in caps.
% 
% form to use if the first word consists of a single letter:
% \IEEEPARstart{A}{demo} file is ....
% 
% form to use if you need the single drop letter followed by
% normal text (unknown if ever used by the IEEE):
% \IEEEPARstart{A}{}demo file is ....
% 
% Some journals put the first two words in caps:
% \IEEEPARstart{T}{his demo} file is ....
% 
% Here we have the typical use of a "T" for an initial drop letter
% and "HIS" in caps to complete the first word.
\IEEEPARstart{A}{cross} the 67 counties located within the Commonwealth of Pennsylvania span 120,000 miles of state and locally owned roads and highways. \cite{PennDOT2024CrashFacts} Accidents are a daily occurrence in today's car-centric infrastructure of America, so tracking and reporting crash rates is a trivial matter. In order to maximize civilian safety and protection statewide, a more in-depth analysis is necessary when considering the daily traffic volume. We are proposing that to spread awareness on road safety and accident prevention, we will be using the Pennsylvania Department of Transportation (PennDOT) car crash data from 2024 in conjunction with PennDOT traffic and mapping data made with ArcGIS to find which areas have the worst crash rate in comparison to its daily traffic volume, what justifies the severity rate of the crashes in these areas, as well as using other factors such as speed limits and seasonal changes can reduce/increase the rate of accidents in these high-volume areas.
% You must have at least 2 lines in the paragraph with the drop letter
% (should never be an issue)


\hfill 
 
\hfill September 15, 2025



% An example of a floating figure using the graphicx package.
% Note that \label must occur AFTER (or within) \caption.
% For figures, \caption should occur after the \includegraphics.
% Note that IEEEtran v1.7 and later has special internal code that
% is designed to preserve the operation of \label within \caption
% even when the captionsoff option is in effect. However, because
% of issues like this, it may be the safest practice to put all your
% \label just after \caption rather than within \caption{}.
%
% Reminder: the "draftcls" or "draftclsnofoot", not "draft", class
% option should be used if it is desired that the figures are to be
% displayed while in draft mode.
%
%\begin{figure}[!t]
%\centering
%\includegraphics[width=2.5in]{myfigure}
% where an .eps filename suffix will be assumed under latex, 
% and a .pdf suffix will be assumed for pdflatex; or what has been declared
% via \DeclareGraphicsExtensions.
%\caption{Simulation results for the network.}
%\label{fig_sim}
%\end{figure}

% Note that the IEEE typically puts floats only at the top, even when this
% results in a large percentage of a column being occupied by floats.


% An example of a double column floating figure using two subfigures.
% (The subfig.sty package must be loaded for this to work.)
% The subfigure \label commands are set within each subfloat command,
% and the \label for the overall figure must come after \caption.
% \hfil is used as a separator to get equal spacing.
% Watch out that the combined width of all the subfigures on a 
% line do not exceed the text width or a line break will occur.
%
%\begin{figure*}[!t]
%\centering
%\subfloat[Case I]{\includegraphics[width=2.5in]{box}%
%\label{fig_first_case}}
%\hfil
%\subfloat[Case II]{\includegraphics[width=2.5in]{box}%
%\label{fig_second_case}}
%\caption{Simulation results for the network.}
%\label{fig_sim}
%\end{figure*}
%
% Note that often IEEE papers with subfigures do not employ subfigure
% captions (using the optional argument to \subfloat[]), but instead will
% reference/describe all of them (a), (b), etc., within the main caption.
% Be aware that for subfig.sty to generate the (a), (b), etc., subfigure
% labels, the optional argument to \subfloat must be present. If a
% subcaption is not desired, just leave its contents blank,
% e.g., \subfloat[].


% An example of a floating table. Note that, for IEEE style tables, the
% \caption command should come BEFORE the table and, given that table
% captions serve much like titles, are usually capitalized except for words
% such as a, an, and, as, at, but, by, for, in, nor, of, on, or, the, to
% and up, which are usually not capitalized unless they are the first or
% last word of the caption. Table text will default to \footnotesize as
% the IEEE normally uses this smaller font for tables.
% The \label must come after \caption as always.
%
%\begin{table}[!t]
%% increase table row spacing, adjust to taste
%\renewcommand{\arraystretch}{1.3}
% if using array.sty, it might be a good idea to tweak the value of
% \extrarowheight as needed to properly center the text within the cells
%\caption{An Example of a Table}
%\label{table_example}
%\centering
%% Some packages, such as MDW tools, offer better commands for making tables
%% than the plain LaTeX2e tabular which is used here.
%\begin{tabular}{|c||c|}
%\hline
%One & Two\\
%\hline
%Three & Four\\
%\hline
%\end{tabular}
%\end{table}


% Note that the IEEE does not put floats in the very first column
% - or typically anywhere on the first page for that matter. Also,
% in-text middle ("here") positioning is typically not used, but it
% is allowed and encouraged for Computer Society conferences (but
% not Computer Society journals). Most IEEE journals/conferences use
% top floats exclusively. 
% Note that, LaTeX2e, unlike IEEE journals/conferences, places
% footnotes above bottom floats. This can be corrected via the
% \fnbelowfloat command of the stfloats package.


\section{Purpose of the project (Jacob Gavin)}
The purpose of this project is to perform a comprehensive analysis of traffic accident data in the 67 counties of Pennsylvania to uncover patterns and underlying trends that contribute to road danger. Although traffic accidents are recorded and summarized by the state, these reports often lack the context of traffic volume, seasonal factors, and environmental factors that can increase risk. By integrating PennDOT crash records with traffic volume data and geospatial tools such as ArcGIS, this project aims to identify road segments and intersections that are disproportionately prone to accidents. 

Beyond simple identification of high-risk areas, this project seeks to examine the contributing variables that define these zones, such as speed limits, lighting conditions, weather and seasonal fluctuations, and temporal patterns such as time of day or weekday vs. weekend. The goal is not only to highlight where crashes are most frequent, but also to determine why these areas are more hazardous, and what features they share. 

Ultimately, the purpose of the project is to generate actionable insights that can inform decision-making by transportation planners, engineers, policymakers, and even the drivers themselves. By translating raw data into meaningful patterns, the project aims to provide a framework for interventions that improve safety, inform infrastructure investment, and reduce injuries and fatalities on Pennsylvania roads.


\section{Beneficiaries of the project (Jacob Gavin)}
This project is intended to benefit drivers in Pennsylvania. Although some of the insights gained from this project may not by themselves be usable by PA drivers, the insights may help to inform state and local policymakers to make decisions that are in the best interest of PA drivers. Ensuring the safety of drivers is a non-partisan issue that is unlikely to be subject to bias by decision makers. For example, suppose it is discovered that there are twice the number of accidents on a particular intersection in the winter months than there are in the summer months. In that case, it may suggest that additional investment needs to be made into winterizing the roads, with perhaps better rock salt coverage, better lighting, or something else. 

The goal of this project will be to uncover the underlying trends in accident data, to see if there are any actions that can be taken to the benefit of Pennsylvania drivers, to ensure their safety on roads throughout the state and throughout the seasons. 
In addition, emergency responders and public health officials could find value in the project’s results. By anticipating accident hotspots and understanding the underlying causes, emergency services can optimize their preparation, potentially reducing response times and improving outcomes for accident victims.

The broader community, including businesses and residents of areas where high accident rates are, will benefit from safer roads and fewer accidents. Reduced crash rates can lead to lower insurance premiums and decreased traffic congestion. Overall, the approach of this project ensures that its benefits extend well beyond individual drivers, supporting the safety of anyone who lives or works in Pennsylvania


\section{Anticipated Technical Challenges (Sam Adebayo)}
Despite the fact that we will be working with a smaller data set, which being either all of PA or certain sections of PA, we will still face challenges in regards to the sizes of our datasets, how we would map things like road size or traffic congestion,  the different variables used for different accident and vehicle types, sorting out location data, etc. Despite this, we can still look at prior papers to see how to move forward, which will be displayed here.

Looking at previous studies, we found that, while complex, solving issues and creating models related to an issue as complex as traffic and accident data is not impossible. One such example comes from \cite{10.1145/3219819.3219922} at Iowa University, where they took data across the state of Iowa and did the following: ``We impose a spatial grid \textit{S }on the study area, where each grid \textit{s(i)} represents a \textit{d × d} square region. For example, if \textit{d = 5km}, the entire state of Iowa can be partitioned into 128 × 64 grids". While this may be a bit more intensive, this is definitely something we can think of doing, considering the fact that our dataset does come with coordinates for the crashes, which we could possibly map in a similar fashion. Deciding where to go with that data should also be easier, since plotting it should give us a better representation of clusters of data and where we should focus. I believe that looking for and finding more research papers related to traffic and accident reports could help us narrow down tactics we could use to interpret and move forward with the data.

The more prominent problem comes in the form of our data and is very apparent when you take a quick glance at it. There are a multitude of variables that we are dealing with, from accident type, to the types of vehicle involved, the time of day and location, number of people injured or involved in the accident, so on, and so forth. We obviously do not plan to use the raw datasets in their entirety, but having to narrow them down for our specific needs will be a large task on its own. Besides this, we may have to edit our final data set because one variable was shown to have no significance, or maybe we missed a variable that could have drastically altered the data for the better. There is a possibility that some of the data we have could be very redundant, which could also hurt any model we decide to make. Overall, we just need to carefully look over our data sets and select which variables we believe will have the largest impact in our model and go from there.


\section{Initial Solution framework and Implementation plan (Logan Camacho)}
Traffic data analysis is an area that our team has not previously explored, so we aim to gain a deeper understanding of the crash data and familiarize ourselves with ArcGIS mapping software to extract traffic risks that contribute to increased crash frequency. Understanding how these crash hotspots occur is imperative to determining where they have the highest likelihood of appearing on the road.

Not only does this involve combing through the datasets for the extractable features and any formatting errors, but also reading the associated PennDOT documentation to further our contextual foundation on the subject matter. With the correlation between tables, table joins, and organizational column additions, this alludes to our team setting the definitive features and response variables for each stage of testing. Once the variables are segmented, we should be entering week 6 of this project as the beginning of model training and looping between variable-model combinations before increasing scale.

To prepare the model for statewide applications, it is in our best interest to begin training using simpler models with fewer iterations before increasing model complexity and region area. This is combined with the variable interpretation previously mentioned to maximize model performance to scale. This performance-at-scale optimization, combined with the preliminary evaluation of models, will mark a positive milestone for week nine. The more time allotted to handling visualization plans for the final presentation, the better off our team will be throughout this project.




\section{Data Overview (Jacob Gavin)}

\subsection{Files and Levels of Observation}

\paragraph{\code{CRASH_2024.csv}}
Crash-level records keyed by \code{CRN}. Contains time, location, environment, roadway context, and severity/count fields. Approx.\
110,814 rows (47 MB).

\paragraph{\code{FLAGS_2024.csv}}
Crash-level engineered binary flags keyed by \code{CRN}. Mirrors \code{CRASH} cardinality. Approx.\ 110,814 rows (56 MB).

\paragraph{\code{ROADWAY_2024.csv}}
Crash–roadway context with one or more rows per \code{CRN} (sequence). Includes lane count, speed limit, access control, route/
segment. Approx.\ 174,250 rows (14 MB).

\paragraph{\code{VEHICLE_2024.csv}}
Unit-level vehicle records keyed by \code{CRN} and \code{UNIT_NUM}. Vehicle type, movement, speed, roles, special usage. Approx.\
197,507 rows (36 MB).

\paragraph{\code{PERSON_2024.csv}}
Person-level records keyed by \code{CRN}, \code{UNIT_NUM}, \code{PERSON_NUM}. Demographics, role, restraint/helmet, injury severity.
Approx.\ 244,174 rows (26 MB).

\paragraph{\code{COMMVEH_2024.csv}}
Commercial vehicle details keyed by \code{CRN}, \code{UNIT_NUM}. GVWR, hazmat, carrier info. Approx.\ 8,717 rows (1.6 MB).

\paragraph{\code{CYCLE_2024.csv}}
Motorcycle-related safety gear and attributes keyed by \code{CRN}, \code{UNIT_NUM}. Approx.\ 3,427 rows (308 KB).

\paragraph{\code{TRAILVEH_2024.csv}}
Trailer details keyed by \code{CRN}, \code{UNIT_NUM} (with trailer sequence). Approx.\ 5,180 rows (312 KB).

\subsection{Key Relationships}
\begin{itemize}
\item Global crash identifier: \code{CRN} links all files.
\item Vehicle joins: \code{CRN} + \code{UNIT_NUM}.
\item Person joins: \code{CRN} + \code{UNIT_NUM} + \code{PERSON_NUM}.
\item Roadway context: one or more records per \code{CRN} (sequence).
\end{itemize}


\section{Schema Summaries (Jacob Gavin)}

\subsection{\code{CRASH_2024.csv} (Crash-Level)}
\begin{itemize}
\item Location/time: \code{DEC_LATITUDE}, \code{DEC_LONGITUDE}, \code{COUNTY}, \code{MUNICIPALITY}, \code{CRASH_MONTH},
\code{DAY_OF_WEEK}, \code{HOUR_OF_DAY}, \code{TIME_OF_DAY}.
\item Environment: \code{WEATHER1}, \code{WEATHER2}, \code{ILLUMINATION}, \code{ROAD_CONDITION}, \code{RDWY_SURF_TYPE_CD}.
\item Control/geometry: \code{TCD_TYPE}, \code{TCD_FUNC_CD}, \code{INTERSECT_TYPE}, \code{INTERSECTION_RELATED},
\code{RELATION_TO_ROAD}, \code{LOCATION_TYPE}.
\item Work zone: \code{WORK_ZONE_IND}, \code{WORK_ZONE_TYPE}, \code{WORK_ZONE_LOC}, \code{WZ_*}.
\item Counts/severity: vehicle/person counts, \code{MAX_SEVERITY_LEVEL}, injury/fatal counts.
\end{itemize}

\subsection{\code{FLAGS_2024.csv} (Crash-Level Flags)}
\begin{itemize}
\item Behaviors/context/crash types: e.g., \code{ALCOHOL_RELATED}, \code{DRUG_RELATED}, \code{DISTRACTED}, \code{SPEEDING_RELATED},
\code{INTERSECTION}, \code{LANE_DEPARTURE}, \code{REAR_END}, \code{LEFT_TURN}, \code{OVERTURNED}, \code{WORK_ZONE}.
\item Outcome-adjacent flags: \code{FATAL}, \code{INJURY}, \code{FATAL_OR_SUSP_SERIOUS_INJ} (exclude as predictors when used as
labels).
\end{itemize}

\subsection{\code{ROADWAY_2024.csv} (Crash--Roadway Context)}
\begin{itemize}
\item Design/operations: \code{LANE_COUNT}, \code{ACCESS_CTRL}, \code{SPEED_LIMIT}, \code{RDWY_ORIENT}, \code{RAMP},
\code{ROAD_OWNER}.
\item Network IDs: \code{ROUTE}, \code{SEGMENT}; enables segment-level aggregation.
\item Cardinality: multiple rows per \code{CRN} possible.
\end{itemize}

\subsection{\code{VEHICLE_2024.csv} (Unit-Level)}
\begin{itemize}[nosep]
\item Attributes: \code{UNIT_TYPE}, \code{VEH_TYPE}, \code{BODY_TYPE}, \code{VEH_COLOR_CD}, \code{MODEL_YR}, \code{COMM_VEH_IND}.
\item Dynamics/context: \code{VEH_MOVEMENT}, \code{TRAVEL_SPD}, \code{IMPACT_POINT}, \code{PRIN_IMP_PT}, \code{TOW_IND},
\code{SPECIAL_USAGE}.
\item Non-motorist interaction indicators included.
\end{itemize}

\subsection{\code{PERSON_2024.csv} (Person-Level)}
\begin{itemize}
\item Demographics/roles: \code{AGE}, \code{SEX}, \code{PERSON_TYPE}, \code{SEAT_POSITION}, \code{NON_MOTORIST}.
\item Safety/condition: \code{RESTRAINT_HELMET}, \code{AIRBAG1}--\code{AIRBAG4}, \code{EJECTION_IND}, \code{INJ_SEVERITY},
\code{TRANSPORTED}.
\end{itemize}

\subsection{\code{COMMVEH_2024.csv}, \code{CYCLE_2024.csv}, \code{TRAILVEH_2024.csv}}
\begin{itemize}
\item Commercial/heavy truck: GVWR, hazmat indicators, carrier identifiers.
\item Motorcycle: helmet and protective gear, engine size.
\item Trailer: presence and type; multiple trailers per unit possible.


\section{Cleaning Plan (Jacob Gavin)}

\subsection{Global Ingestion and Normalization}
\begin{enumerate}
\item Read CSVs with consistent encoding; trim whitespace; preserve provided column names.
\item Enforce types:
    \begin{itemize}
      \item Identifiers: \code{CRN} as string-like key; \code{UNIT_NUM}, \code{PERSON_NUM} as integers.
      \item Categorical: codes (e.g., \code{TCD_TYPE}, \code{ILLUMINATION}, \code{UNIT_TYPE}) as categorical strings.
      \item Numeric: counts, speeds, limits, lane counts to numeric with coercion.
      \item Binary flags: coerce to 0/1 integers.
    \end{itemize}
\item Standardize missing values: convert blanks and placeholders (e.g., \code{NA}, \code{NULL}) to NA.
\item Deduplicate: ensure \code{CRN} uniqueness at crash level; drop exact duplicates elsewhere.
\item Geospatial checks: validate latitude/longitude ranges for Pennsylvania; flag out-of-bounds.
\item Temporal checks: verify \code{CRASH_YEAR}=2024; standardize month/day/hour fields.
\item Category harmonization: document codebooks for categorical values.
\end{enumerate}
% if have a single appendix:
%\appendix[Proof of the Zonklar Equations]
% or
%\appendix  % for no appendix heading
% do not use \section anymore after \appendix, only \section*
% is possibly needed

% use appendices with more than one appendix
% then use \section to start each appendix
% you must declare a \section before using any
% \subsection or using \label (\appendices by itself
% starts a section numbered zero.)
%


\appendices
%\section{}
%Appendix one text goes here.

% you can choose not to have a title for an appendix
% if you want by leaving the argument blank
%\section{}
%Appendix two text goes here.


% use section* for acknowledgment
%\section*{Acknowledgment}





% Can use something like this to put references on a page
% by themselves when using endfloat and the captionsoff option.
\ifCLASSOPTIONcaptionsoff
  \newpage
\fi



% trigger a \newpage just before the given reference
% number - used to balance the columns on the last page
% adjust value as needed - may need to be readjusted if
% the document is modified later
%\IEEEtriggeratref{8}
% The "triggered" command can be changed if desired:
%\IEEEtriggercmd{\enlargethispage{-5in}}

% references section

% can use a bibliography generated by BibTeX as a .bbl file
% BibTeX documentation can be easily obtained at:
% http://mirror.ctan.org/biblio/bibtex/contrib/doc/
% The IEEEtran BibTeX style support page is at:
% http://www.michaelshell.org/tex/ieeetran/bibtex/
%\bibliographystyle{IEEEtran}

% argument is your BibTeX string definitions and bibliography database(s)
\bibliographystyle{IEEEtran}
% argument is your BibTeX string definitions and bibliography database(s)
\bibliography{references}

%
% <OR> manually copy in the resultant .bbl file
% set second argument of \begin to the number of references
% (used to reserve space for the reference number labels box)


% biography section
% 
% If you have an EPS/PDF photo (graphicx package needed) extra braces are
% needed around the contents of the optional argument to biography to prevent
% the LaTeX parser from getting confused when it sees the complicated
% \includegraphics command within an optional argument. (You could create
% your own custom macro containing the \includegraphics command to make things
% simpler here.)
%\begin{IEEEbiography}[{\includegraphics[width=1in,height=1.25in,clip,keepaspectratio]{mshell}}]{Michael Shell}
% or if you just want to reserve a space for a photo:

%\begin{IEEEbiographynophoto}{Logan Camacho}
%Biography text here.
%\end{IEEEbiographynophoto}

% if you will not have a photo at all:
%\begin{IEEEbiographynophoto}{Jacob Gavin}
%Biography text here.
%\end{IEEEbiographynophoto}

% insert where needed to balance the two columns on the last page with
% biographies
%\newpage

%\begin{IEEEbiographynophoto}{Sam Adebayo}
%Biography text here.
%\end{IEEEbiographynophoto}

% You can push biographies down or up by placing
% a \vfill before or after them. The appropriate
% use of \vfill depends on what kind of text is
% on the last page and whether or not the columns
% are being equalized.

%\vfill

% Can be used to pull up biographies so that the bottom of the last one
% is flush with the other column.
%\enlargethispage{-5in}



% that's all folks
\end{document}


